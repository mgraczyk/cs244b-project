Figures 1 through 6 show the latencies of the two systems under each of the nine tasks and
experiments. Figure 7 shows the average, 99\%, and 99.9\% latencies for the mixed conflicting
read-write task of each experiment.

The results show that ZooKeeper has much lower latency for reads than writes. In addition, read
latency during conflicting writes is significantly higher than Safari. We believe that ZooKeeper's
read latency could be improved through the use of UDP instead of TCP, C++ instead of Java, and with
clients sending requests to multiple servers and awaiting the first response rather than always
using the same server.

We can see from Figure 7 that tail latency is fairly good in both systems. In both real world
settings, ZooKeeper's 99.9\% latency is no more than 2x its average latency. Safari has must lower
tail latency in the two datacenter deployment because read requests can complete successfully with
no round trips outside of the west coast datacenter. Even in the three datacenter settings, Safari
has low tail latency, only $\approx 5\%$ greater than the average latency. This is because reads
complete successfully as soon as the client receives any two responses, so slow responses from any
one data center do not matter.

Figure 7 also shows the surprising result that ZooKeeper had worse latency in the two datacenter
deployment than in the three datacenter deployment. We believe this is probably caused by the west
coast client accessing the east coast server for reads, although this claim should be investigated
further.

We claimed that Safari was designed to minimize latency. Indeed, the Figures clearly show that
Safari has more consistent and much lower tail latency than ZooKeeper. This is not surprising
considering the Safari's shortcomings. However, the local experiment in particular demonstrates that
ZooKeeper's latency could be drastically improved by changes to the implementation.

We also found that ZooKeeper performed inconsistently across runs of the same experiment. Typically
each run would take just a minute or two. Roughly one third of the time, ZooKeeper would crawl at a
pace such that it would have taken almost an hour to finish the experiment. This may have been
caused by clients selecting distant servers from which to read, and could probably have been
resolved through reconfiguration, tuning, or with different client software. However, we were
surprised that performance was so inconsistent in a seemingly typical deployment.

Overall we found ZooKeeper deployment to be fairly simple. All of the code necessary to download and
run ZooKeeper on EC2 can be found in roughly 5 lines of shell script and 15 lines of ZooKeeper
configuration. Although we had to manually tell each ZooKeeper server the IP addresses of all other
servers, this could be made less painful by spending more and purchasing long-lived IPs instead of
the transient ones we used, or using managed DNS. Safari was of course also easy to configure.
Athough the current implementation accepts a list of peer servers, servers never send messages to
one another. The only necessary configuration is to choose a UDP port on which the servers listen
for messages.

\begin{figure}[h]
  \caption{Zookeeper Latency on a Single Machine}
  \centering
  \includegraphics[width=0.5\textwidth]{../results/local/zookeeper.pdf}
\end{figure}

\begin{figure}[h]
  \caption{Safari Latency on a Single Machine}
  \centering
  \includegraphics[width=0.5\textwidth]{../results/local/safari.pdf}
\end{figure}


\begin{figure}[h]
  \caption{Zookeeper Latency Distributed Across Two Datacenters}
  \centering
  \includegraphics[width=0.5\textwidth]{../results/exp1/zookeeper.pdf}
\end{figure}

\begin{figure}[h]
  \caption{Safari Latency Distributed Across Two Datacenters}
  \centering
  \includegraphics[width=0.5\textwidth]{../results/exp1/safari.pdf}
\end{figure}

\begin{figure}[h]
  \caption{Zookeeper Latency Distributed Across Three Datacenters}
  \centering
  \includegraphics[width=0.5\textwidth]{../results/exp2/zookeeper.pdf}
\end{figure}

\begin{figure}[h]
  \caption{Safari Latency Distributed Across Three Datacenters}
  \centering
  \includegraphics[width=0.5\textwidth]{../results/exp2/safari.pdf}
\end{figure}


\begin{figure}[h]
  \caption{Mixed Read-Write Latency Statistics (ms)}
  \begin{center}
    \begin{tabular}{| l | l | l | l |}
    \hline
    & Average & 99\% & 99.9\% \\ \hline
    Local ZooKeeper & 13.414 & 26.090 & 33.073 \\ \hline
    Local Safari & 0.307 & 0.441 & 1.341 \\ \hline\hline
    2 Datacenter ZooKeeper & 148.230 & 272.393 & 279.877 \\ \hline
    2 Datacenter Safari & 0.728 & 1.052 & 1.439 \\ \hline\hline
    3 Datacenters ZooKeeper & 76.873 & 93.481 & 94.716 \\ \hline
    3 Datacenters Safari & 45.981 & 46.391 & 47.748 \\
    \hline
    \end{tabular}
  \end{center}
\end{figure}
