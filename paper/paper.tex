% IEEEtran V1.7 and later provides for these CLASSINPUT macros to allow the
% user to reprogram some IEEEtran.cls defaults if needed. These settings
% override the internal defaults of IEEEtran.cls regardless of which class
% options are used. Do not use these unless you have good reason to do so as
% they can result in nonIEEE compliant documents. User beware. ;)
%
%\newcommand{\CLASSINPUTbaselinestretch}{1.0} % baselinestretch
%\newcommand{\CLASSINPUTinnersidemargin}{1in} % inner side margin
%\newcommand{\CLASSINPUToutersidemargin}{1in} % outer side margin
%\newcommand{\CLASSINPUTtoptextmargin}{1in}   % top text margin
%\newcommand{\CLASSINPUTbottomtextmargin}{1in}% bottom text margin




%
\documentclass[10pt,journal,compsoc,conference]{IEEEtran}
% If IEEEtran.cls has not been installed into the LaTeX system files,
% manually specify the path to it like:
% \documentclass[10pt,journal,compsoc]{../sty/IEEEtran}


% For Computer Society journals, IEEEtran defaults to the use of
% Palatino/Palladio as is done in IEEE Computer Society journals.
% To go back to Times Roman, you can use this code:
%\renewcommand{\rmdefault}{ptm}\selectfont





% Some very useful LaTeX packages include:
% (uncomment the ones you want to load)



% *** MISC UTILITY PACKAGES ***
%
%\usepackage{ifpdf}
% Heiko Oberdiek's ifpdf.sty is very useful if you need conditional
% compilation based on whether the output is pdf or dvi.
% usage:
% \ifpdf
%   % pdf code
% \else
%   % dvi code
% \fi
% The latest version of ifpdf.sty can be obtained from:
% http://www.ctan.org/pkg/ifpdf
% Also, note that IEEEtran.cls V1.7 and later provides a builtin
% \ifCLASSINFOpdf conditional that works the same way.
% When switching from latex to pdflatex and vice-versa, the compiler may
% have to be run twice to clear warning/error messages.






% *** CITATION PACKAGES ***
%
\ifCLASSOPTIONcompsoc
  % The IEEE Computer Society needs nocompress option
  % requires cite.sty v4.0 or later (November 2003)
  \usepackage[nocompress]{cite}
\else
  % normal IEEE
  \usepackage{cite}
\fi
% cite.sty was written by Donald Arseneau
% V1.6 and later of IEEEtran pre-defines the format of the cite.sty package
% \cite{} output to follow that of the IEEE. Loading the cite package will
% result in citation numbers being automatically sorted and properly
% "compressed/ranged". e.g., [1], [9], [2], [7], [5], [6] without using
% cite.sty will become [1], [2], [5]--[7], [9] using cite.sty. cite.sty's
% \cite will automatically add leading space, if needed. Use cite.sty's
% noadjust option (cite.sty V3.8 and later) if you want to turn this off
% such as if a citation ever needs to be enclosed in parenthesis.
% cite.sty is already installed on most LaTeX systems. Be sure and use
% version 5.0 (2009-03-20) and later if using hyperref.sty.
% The latest version can be obtained at:
% http://www.ctan.org/pkg/cite
% The documentation is contained in the cite.sty file itself.
%
% Note that some packages require special options to format as the Computer
% Society requires. In particular, Computer Society  papers do not use
% compressed citation ranges as is done in typical IEEE papers
% (e.g., [1]-[4]). Instead, they list every citation separately in order
% (e.g., [1], [2], [3], [4]). To get the latter we need to load the cite
% package with the nocompress option which is supported by cite.sty v4.0
% and later.





% *** GRAPHICS RELATED PACKAGES ***
%
\ifCLASSINFOpdf
  \usepackage[pdftex]{graphicx}
  % declare the path(s) where your graphic files are
  % \graphicspath{{../pdf/}{../jpeg/}}
  % and their extensions so you won't have to specify these with
  % every instance of \includegraphics
  \DeclareGraphicsExtensions{.pdf,.jpeg,.png}
\else
  % or other class option (dvipsone, dvipdf, if not using dvips). graphicx
  % will default to the driver specified in the system graphics.cfg if no
  % driver is specified.
  % \usepackage[dvips]{graphicx}
  % declare the path(s) where your graphic files are
  % \graphicspath{{../eps/}}
  % and their extensions so you won't have to specify these with
  % every instance of \includegraphics
  % \DeclareGraphicsExtensions{.eps}
\fi
% graphicx was written by David Carlisle and Sebastian Rahtz. It is
% required if you want graphics, photos, etc. graphicx.sty is already
% installed on most LaTeX systems. The latest version and documentation
% can be obtained at:
% http://www.ctan.org/pkg/graphicx
% Another good source of documentation is "Using Imported Graphics in
% LaTeX2e" by Keith Reckdahl which can be found at:
% http://www.ctan.org/pkg/epslatex
%
% latex, and pdflatex in dvi mode, support graphics in encapsulated
% postscript (.eps) format. pdflatex in pdf mode supports graphics
% in .pdf, .jpeg, .png and .mps (metapost) formats. Users should ensure
% that all non-photo figures use a vector format (.eps, .pdf, .mps) and
% not a bitmapped formats (.jpeg, .png). The IEEE frowns on bitmapped formats
% which can result in "jaggedy"/blurry rendering of lines and letters as
% well as large increases in file sizes.
%
% You can find documentation about the pdfTeX application at:
% http://www.tug.org/applications/pdftex





% *** MATH PACKAGES ***
%
\usepackage{amsmath, amsthm, amsfonts, amsbsy, amssymb}
\usepackage{mathtools}
\usepackage{float}
% A popular package from the American Mathematical Society that provides
% many useful and powerful commands for dealing with mathematics.
%
% Note that the amsmath package sets \interdisplaylinepenalty to 10000
% thus preventing page breaks from occurring within multiline equations. Use:
%\interdisplaylinepenalty=2500
% after loading amsmath to restore such page breaks as IEEEtran.cls normally
% does. amsmath.sty is already installed on most LaTeX systems. The latest
% version and documentation can be obtained at:
% http://www.ctan.org/pkg/amsmath





% *** SPECIALIZED LIST PACKAGES ***
%\usepackage{acronym}
% acronym.sty was written by Tobias Oetiker. This package provides tools for
% managing documents with large numbers of acronyms. (You don't *have* to
% use this package - unless you have a lot of acronyms, you may feel that
% such package management of them is bit of an overkill.)
% Do note that the acronym environment (which lists acronyms) will have a
% problem when used under IEEEtran.cls because acronym.sty relies on the
% description list environment - which IEEEtran.cls has customized for
% producing IEEE style lists. A workaround is to declared the longest
% label width via the IEEEtran.cls \IEEEiedlistdecl global control:
%
% \renewcommand{\IEEEiedlistdecl}{\IEEEsetlabelwidth{SONET}}
% \begin{acronym}
%
% \end{acronym}
% \renewcommand{\IEEEiedlistdecl}{\relax}% remember to reset \IEEEiedlistdecl
%
% instead of using the acronym environment's optional argument.
% The latest version and documentation can be obtained at:
% http://www.ctan.org/pkg/acronym


%\usepackage{algorithmic}
% algorithmic.sty was written by Peter Williams and Rogerio Brito.
% This package provides an algorithmic environment fo describing algorithms.
% You can use the algorithmic environment in-text or within a figure
% environment to provide for a floating algorithm. Do NOT use the algorithm
% floating environment provided by algorithm.sty (by the same authors) or
% algorithm2e.sty (by Christophe Fiorio) as the IEEE does not use dedicated
% algorithm float types and packages that provide these will not provide
% correct IEEE style captions. The latest version and documentation of
% algorithmic.sty can be obtained at:
% http://www.ctan.org/pkg/algorithms
% Also of interest may be the (relatively newer and more customizable)
% algorithmicx.sty package by Szasz Janos:
% http://www.ctan.org/pkg/algorithmicx




% *** ALIGNMENT PACKAGES ***
%
%\usepackage{array}
% Frank Mittelbach's and David Carlisle's array.sty patches and improves
% the standard LaTeX2e array and tabular environments to provide better
% appearance and additional user controls. As the default LaTeX2e table
% generation code is lacking to the point of almost being broken with
% respect to the quality of the end results, all users are strongly
% advised to use an enhanced (at the very least that provided by array.sty)
% set of table tools. array.sty is already installed on most systems. The
% latest version and documentation can be obtained at:
% http://www.ctan.org/pkg/array


%\usepackage{mdwmath}
%\usepackage{mdwtab}
% Also highly recommended is Mark Wooding's extremely powerful MDW tools,
% especially mdwmath.sty and mdwtab.sty which are used to format equations
% and tables, respectively. The MDWtools set is already installed on most
% LaTeX systems. The lastest version and documentation is available at:
% http://www.ctan.org/pkg/mdwtools


% IEEEtran contains the IEEEeqnarray family of commands that can be used to
% generate multiline equations as well as matrices, tables, etc., of high
% quality.


%\usepackage{eqparbox}
% Also of notable interest is Scott Pakin's eqparbox package for creating
% (automatically sized) equal width boxes - aka "natural width parboxes".
% Available at:
% http://www.ctan.org/pkg/eqparbox




% *** SUBFIGURE PACKAGES ***
%\ifCLASSOPTIONcompsoc
%  \usepackage[caption=false,font=footnotesize,labelfont=sf,textfont=sf]{subfig}
%\else
%  \usepackage[caption=false,font=footnotesize]{subfig}
%\fi
% subfig.sty, written by Steven Douglas Cochran, is the modern replacement
% for subfigure.sty, the latter of which is no longer maintained and is
% incompatible with some LaTeX packages including fixltx2e. However,
% subfig.sty requires and automatically loads Axel Sommerfeldt's caption.sty
% which will override IEEEtran.cls' handling of captions and this will result
% in non-IEEE style figure/table captions. To prevent this problem, be sure
% and invoke subfig.sty's "caption=false" package option (available since
% subfig.sty version 1.3, 2005/06/28) as this is will preserve IEEEtran.cls
% handling of captions.
% Note that the Computer Society format requires a sans serif font rather
% than the serif font used in traditional IEEE formatting and thus the need
% to invoke different subfig.sty package options depending on whether
% compsoc mode has been enabled.
%
% The latest version and documentation of subfig.sty can be obtained at:
% http://www.ctan.org/pkg/subfig




% *** FLOAT PACKAGES ***
%
%\usepackage{fixltx2e}
% fixltx2e, the successor to the earlier fix2col.sty, was written by
% Frank Mittelbach and David Carlisle. This package corrects a few problems
% in the LaTeX2e kernel, the most notable of which is that in current
% LaTeX2e releases, the ordering of single and double column floats is not
% guaranteed to be preserved. Thus, an unpatched LaTeX2e can allow a
% single column figure to be placed prior to an earlier double column
% figure.
% Be aware that LaTeX2e kernels dated 2015 and later have fixltx2e.sty's
% corrections already built into the system in which case a warning will
% be issued if an attempt is made to load fixltx2e.sty as it is no longer
% needed.
% The latest version and documentation can be found at:
% http://www.ctan.org/pkg/fixltx2e


%\usepackage{stfloats}
% stfloats.sty was written by Sigitas Tolusis. This package gives LaTeX2e
% the ability to do double column floats at the bottom of the page as well
% as the top. (e.g., "\begin{figure*}[!b]" is not normally possible in
% LaTeX2e). It also provides a command:
%\fnbelowfloat
% to enable the placement of footnotes below bottom floats (the standard
% LaTeX2e kernel puts them above bottom floats). This is an invasive package
% which rewrites many portions of the LaTeX2e float routines. It may not work
% with other packages that modify the LaTeX2e float routines. The latest
% version and documentation can be obtained at:
% http://www.ctan.org/pkg/stfloats
% Do not use the stfloats baselinefloat ability as the IEEE does not allow
% \baselineskip to stretch. Authors submitting work to the IEEE should note
% that the IEEE rarely uses double column equations and that authors should try
% to avoid such use. Do not be tempted to use the cuted.sty or midfloat.sty
% packages (also by Sigitas Tolusis) as the IEEE does not format its papers in
% such ways.
% Do not attempt to use stfloats with fixltx2e as they are incompatible.
% Instead, use Morten Hogholm'a dblfloatfix which combines the features
% of both fixltx2e and stfloats:
%
% \usepackage{dblfloatfix}
% The latest version can be found at:
% http://www.ctan.org/pkg/dblfloatfix


%\ifCLASSOPTIONcaptionsoff
%  \usepackage[nomarkers]{endfloat}
% \let\MYoriglatexcaption\caption
% \renewcommand{\caption}[2][\relax]{\MYoriglatexcaption[#2]{#2}}
%\fi
% endfloat.sty was written by James Darrell McCauley, Jeff Goldberg and
% Axel Sommerfeldt. This package may be useful when used in conjunction with
% IEEEtran.cls'  captionsoff option. Some IEEE journals/societies require that
% submissions have lists of figures/tables at the end of the paper and that
% figures/tables without any captions are placed on a page by themselves at
% the end of the document. If needed, the draftcls IEEEtran class option or
% \CLASSINPUTbaselinestretch interface can be used to increase the line
% spacing as well. Be sure and use the nomarkers option of endfloat to
% prevent endfloat from "marking" where the figures would have been placed
% in the text. The two hack lines of code above are a slight modification of
% that suggested by in the endfloat docs (section 8.4.1) to ensure that
% the full captions always appear in the list of figures/tables - even if
% the user used the short optional argument of \caption[]{}.
% IEEE papers do not typically make use of \caption[]'s optional argument,
% so this should not be an issue. A similar trick can be used to disable
% captions of packages such as subfig.sty that lack options to turn off
% the subcaptions:
% For subfig.sty:
% \let\MYorigsubfloat\subfloat
% \renewcommand{\subfloat}[2][\relax]{\MYorigsubfloat[]{#2}}
% However, the above trick will not work if both optional arguments of
% the \subfloat command are used. Furthermore, there needs to be a
% description of each subfigure *somewhere* and endfloat does not add
% subfigure captions to its list of figures. Thus, the best approach is to
% avoid the use of subfigure captions (many IEEE journals avoid them anyway)
% and instead reference/explain all the subfigures within the main caption.
% The latest version of endfloat.sty and its documentation can obtained at:
% http://www.ctan.org/pkg/endfloat
%
% The IEEEtran \ifCLASSOPTIONcaptionsoff conditional can also be used
% later in the document, say, to conditionally put the References on a
% page by themselves.



% *** PDF, URL AND HYPERLINK PACKAGES ***
%
\usepackage{url}
% url.sty was written by Donald Arseneau. It provides better support for
% handling and breaking URLs. url.sty is already installed on most LaTeX
% systems. The latest version and documentation can be obtained at:
% http://www.ctan.org/pkg/url
% Basically, \url{my_url_here}.


% NOTE: PDF thumbnail features are not required in IEEE papers
%       and their use requires extra complexity and work.
%\ifCLASSINFOpdf
%  \usepackage[pdftex]{thumbpdf}
%\else
%  \usepackage[dvips]{thumbpdf}
%\fi
% thumbpdf.sty and its companion Perl utility were written by Heiko Oberdiek.
% It allows the user a way to produce PDF documents that contain fancy
% thumbnail images of each of the pages (which tools like acrobat reader can
% utilize). This is possible even when using dvi->ps->pdf workflow if the
% correct thumbpdf driver options are used. thumbpdf.sty incorporates the
% file containing the PDF thumbnail information (filename.tpm is used with
% dvips, filename.tpt is used with pdftex, where filename is the base name of
% your tex document) into the final ps or pdf output document. An external
% utility, the thumbpdf *Perl script* is needed to make these .tpm or .tpt
% thumbnail files from a .ps or .pdf version of the document (which obviously
% does not yet contain pdf thumbnails). Thus, one does a:
%
% thumbpdf filename.pdf
%
% to make a filename.tpt, and:
%
% thumbpdf --mode dvips filename.ps
%
% to make a filename.tpm which will then be loaded into the document by
% thumbpdf.sty the NEXT time the document is compiled (by pdflatex or
% latex->dvips->ps2pdf). Users must be careful to regenerate the .tpt and/or
% .tpm files if the main document changes and then to recompile the
% document to incorporate the revised thumbnails to ensure that thumbnails
% match the actual pages. It is easy to forget to do this!
%
% Unix systems come with a Perl interpreter. However, MS Windows users
% will usually have to install a Perl interpreter so that the thumbpdf
% script can be run. The Ghostscript PS/PDF interpreter is also required.
% See the thumbpdf docs for details. The latest version and documentation
% can be obtained at.
% http://www.ctan.org/pkg/thumbpdf


% NOTE: PDF hyperlink and bookmark features are not required in IEEE
%       papers and their use requires extra complexity and work.
\newcommand\MYhyperrefoptions{bookmarks=true,bookmarksnumbered=true,
pdfpagemode={UseOutlines},plainpages=false,pdfpagelabels=true,
colorlinks=true,linkcolor={black},citecolor={black},urlcolor={black},
pdftitle={Tail Latency in ZooKeeper and a Simple Reimplementation},
pdfsubject={Distributed Systems},
pdfauthor={Graczyk},
pdfkeywords={distributed systems,zookeeper} }
%\ifCLASSINFOpdf
%\usepackage[\MYhyperrefoptions,pdftex]{hyperref}
%\else
%\usepackage[\MYhyperrefoptions,breaklinks=true,dvips]{hyperref}
%\usepackage{breakurl}
%\fi
% One significant drawback of using hyperref under DVI output is that the
% LaTeX compiler cannot break URLs across lines or pages as can be done
% under pdfLaTeX's PDF output via the hyperref pdftex driver. This is
% probably the single most important capability distinction between the
% DVI and PDF output. Perhaps surprisingly, all the other PDF features
% (PDF bookmarks, thumbnails, etc.) can be preserved in
% .tex->.dvi->.ps->.pdf workflow if the respective packages/scripts are
% loaded/invoked with the correct driver options (dvips, etc.).
% As most IEEE papers use URLs sparingly (mainly in the references), this
% may not be as big an issue as with other publications.
%
% That said, Vilar Camara Neto created his breakurl.sty package which
% permits hyperref to easily break URLs even in dvi mode.
% Note that breakurl, unlike most other packages, must be loaded
% AFTER hyperref. The latest version of breakurl and its documentation can
% be obtained at:
% http://www.ctan.org/pkg/breakurl
% breakurl.sty is not for use under pdflatex pdf mode.
%
% The advanced features offer by hyperref.sty are not required for IEEE
% submission, so users should weigh these features against the added
% complexity of use.
% The package options above demonstrate how to enable PDF bookmarks
% (a type of table of contents viewable in Acrobat Reader) as well as
% PDF document information (title, subject, author and keywords) that is
% viewable in Acrobat reader's Document_Properties menu. PDF document
% information is also used extensively to automate the cataloging of PDF
% documents. The above set of options ensures that hyperlinks will not be
% colored in the text and thus will not be visible in the printed page,
% but will be active on "mouse over". USING COLORS OR OTHER HIGHLIGHTING
% OF HYPERLINKS CAN RESULT IN DOCUMENT REJECTION BY THE IEEE, especially if
% these appear on the "printed" page. IF IN DOUBT, ASK THE RELEVANT
% SUBMISSION EDITOR. You may need to add the option hypertexnames=false if
% you used duplicate equation numbers, etc., but this should not be needed
% in normal IEEE work.
% The latest version of hyperref and its documentation can be obtained at:
% http://www.ctan.org/pkg/hyperref





% *** Do not adjust lengths that control margins, column widths, etc. ***
% *** Do not use packages that alter fonts (such as pslatex).         ***
% There should be no need to do such things with IEEEtran.cls V1.6 and later.
% (Unless specifically asked to do so by the journal or conference you plan
% to submit to, of course. )


% correct bad hyphenation here
\hyphenation{op-tical net-works semi-conduc-tor}

% Macro definitions
\newcommand{\RR}{\mathbb{R}}
\newcommand{\partialof}[2]{\frac{\partial #2}{\partial #1}}
\newcommand{\norm}[1]{\left\lVert#1\right\rVert}
\newcommand{\iu}{{i\mkern1mu}}
\newcommand{\argmin}{\operatornamewithlimits{argmin}\,}
\newcommand{\argmax}{\operatornamewithlimits{argmax}\,}
\newcommand{\Expect}{{\rm I\kern-.3em E}}
\newcommand*\diff{\mathop{}\!\mathrm{d}}


\begin{document}
%
% paper title
% Titles are generally capitalized except for words such as a, an, and, as,
% at, but, by, for, in, nor, of, on, or, the, to and up, which are usually
% not capitalized unless they are the first or last word of the title.
% Linebreaks \\ can be used within to get better formatting as desired.
% Do not put math or special symbols in the title.
\title{Tail Latency in ZooKeeper and a Simple Reimplementation}
%
%
% author names
% note positions of commas and nonbreaking spaces ( ~ ) LaTeX will not break
% a structure at a ~ so this keeps an author's name from being broken across
% two lines.
\author{Michael~Graczyk}

% The paper headers
\markboth{CS240b Final Project}%
{Graczyk: Tail Latency in ZooKeeper and a Simple Reimplementation}
% The only time the second header will appear is for the odd numbered pages
% after the title page when using the twoside option.


% for Computer Society papers, we must declare the abstract and index terms
% PRIOR to the title within the \IEEEtitleabstractindextext IEEEtran
% command as these need to go into the title area created by \maketitle.
% As a general rule, do not put math, special symbols or citations
% in the abstract or keywords.
\IEEEtitleabstractindextext{%
\begin{abstract}
ZooKeeper [1] is a commonly used service for coordinating distributed applications.
ZooKeeper uses leader-based atomic broadcast for writes, so that all state modifications
are globally totally ordered, but allows stale reads from any server for high read availability.
This design trades high read throughput for potentially high write latency.
Although [1] presents average request latency, tail latency is often far more important in real world applications [2] [3].
\par
As a baseline, we also implemented a system called Safari with the same API as ZooKeeper, but
specifically designed for lower worst-case write latency.
We examine the two systems' latency characteristics in a single-machine and two realistic production environments.
We also offer explanations for their performance differences and design tradeoffs, as well as some
  comments on practically deploying ZooKeeper and Safari.
\end{abstract}
}

% make the title area
\maketitle


% To allow for easy dual compilation without having to reenter the
% abstract/keywords data, the \IEEEtitleabstractindextext text will
% not be used in maketitle, but will appear (i.e., to be "transported")
% here as \IEEEdisplaynontitleabstractindextext when compsoc mode
% is not selected <OR> if conference mode is selected - because compsoc
% conference papers position the abstract like regular (non-compsoc)
% papers do!
\IEEEdisplaynontitleabstractindextext
% \IEEEdisplaynontitleabstractindextext has no effect when using
% compsoc under a non-conference mode.


% For peerreview papers, this IEEEtran command inserts a page break and
% creates the second title. It will be ignored for other modes.
\IEEEpeerreviewmaketitle


\section{Introduction}\label{sec:introduction}
Distributed systems are notoriously complicated to build. ZooKeeper has become popular largely due
to its use as a building block to make other distributed systems less complicated. Developers rely
on ZooKeeper's strong write consistency and high read availibility to offload design complexity and
make their own systems simpler. For example, FaRM [4] uses ZooKeeper to manage configuration so that the
authors can focus on high performance design and make rare, hairy failure recovery simple. FaRM
avoids using ZooKeeper in the critical path, presumably because of ZooKeeper's relatively high
latency. Apache Kafka [5], a real-time stream processing platform, uses ZooKeeper to manage cluster
membership, leadership, and various other metadata. As with FaRM, Kafka avoids using ZooKeeper in
the system's critical path.

Despite ZooKeeper's common usage and many benchmarks reporting its average case latency [6], there
do not seem to be any reports of the system's tail latency under load. Tail latency is amongst the
most important performance characteristics in many real-world systems because these systems are
composed of many interdependent components. With high "fan out", even rare spikes in latency amongst
a small number of components can cause the overall system to respond slowly on average.  Accurate
characterization of ZooKeeper's worst case latency is important for potential application developers
to determine how best to fit ZooKeeper into high fan-out systems.

Certain aspects of ZooKeeper's design could lead to occassionaly high request latency. Some members
of the community have suggested that latency is primarily determined by the time spent spent by
followers fetching data from the leader [7]. This characteristic suggests that ZooKeeper could
potential have lower latency with a leaderless design.

In order to compare ZooKeeper to a low latency baseline, we implemented a system called Safari.
Safari aims to provide the same consistency guarantees as ZooKeeper with lower tail latency, while
sacrificing read and write throughput, availability during network partitions, and features. Our
system is currently incomplete and does not provide linearizability as intended, but serves as an
optimistic lower bound on latency that could be achieved for any system with ZooKeeper's API and
consistency.


\section{Safari}
Although our aspirations for Safari were higher, the system as currently implemented is extremely
basic. Each server stores a copy of the ZooKeeper tree-of-znodes data structure. Clients modify
state by sending a modification requests to all servers. Clients read state by requesting it from
all servers, and returning the data to the client application once a majority of servers have
returned the same value. All communication is done using UDP based message passing. That is, there
is no connection state. Messages are currently restricted to fit in a single UDP packet, so znode
data must be no greater than than $\approx 60kB$. We have not implemented watches or sequential
znodes, but these would be easy to add the the existing system.

Although we believe the system offers linearizable state changes, it is currently useless in
practice. The system can quickly become unavailable when multiple clients make state modifications
concurrently, especially when latencies are large. Although the system remained available during our
local experiment, it frequently halted during our real-world deployment experiments. As a result,
\textbf{Safari's latencies in these experiments should be interpreted as a lower bound on any
ZooKeeper-like system.} Still, we believe that the reported read latencies are achievable because in
most cases (ie, the network is not partitioned and servers have not failed) reads could behave
exactly as they do in the current implementation even with changes to make the system more
available.

We had hoped to implement a leaderless consensus algorithm like AllConcur [10] to keep the system
available and automatically resolve conflicts while preserving low latency. This would also decrease
read latency because clients could deliever results to applications after receiving just one
successful response, rather than waiting for a majority. However, we have not completed this
implementation in yet.

We defer most discussion of implementation details to a video describing the system [8] and the
source code [9]. Additional message passing would be required to resolve these conflicts when they
are detected, so the system's latency provides a loose lower bound on what could be expected from a
ZooKeeper implementation with the same consistency.


\section{Experiment Setup}
We tested ZooKeeper and Safari's latencies using three tasks under three experimental settings. In
order to measure the relative overhead of the software implementations themselves, rather than the
algorithms and their messaging latency, we first ran the client and servers on a single 2017 Macbook
Pro. For the next two experiments, we deployed ZooKeeper and Safari on AWS EC2 $m3.xlarge$ instances
with attached SSDs.

In our second experiment, we ran the systems with two servers in the same west coast data center,
and a third server on the east coast. The client was also located in the west coast data center.
This experimental setup offers resiliance to the loss of a single machine in the west coast data
center, or the entire east coast data center. In principle, a quorum system with this configuration
should have low latency because the two colocated servers could commit writes as a quorum with low
latency while the east coast data center operates as a follower.

For the third and final experiment, we ran servers on three different data centers in Northern
California, Oregon, and Virginia. The client also ran in Northern California. This setup offers
resiliance to the loss of any data center, but any robust system must pass messages between
datacenters to commit writes. Under these circumstances, systems which minimizes total sequential
messages should have the best latencies.

Each experiment consisted of three tasks run sequentially. In the first task, a single client
creates 5 keys with 1000 bytes of data and reads data from a randomly selected key 1000 times. This
tasks tests the systems' best case read latencies. In the second task, a single client creates 5
keys and writes 1000 bytes to a randomly selected key 1000 times. Like the first task, this one
tests the sytems' best case write latencies.

The third and most important tasks consists of mixed, conflicting reads and writes.  We create 5
keys with 1000 bytes of random data, then start 6 concurrent clients. Each client does the following
as fast as possible.
\begin{itemize}
  \item Select a "read key" at random and read the data from this key.
  \item Select a "write key" at random.
  \item If the selected "read key" is even, write all but the last byte read to the "write key".
  \item If the selected "read key" is odd, append a byte to the data and write it to the "write key".
  \item Repeat the process indefinitely or 1000 times if this is the first client.
\end{itemize}

The above process simulates heavy read-write contention and should stress ZooKeeper because of its
centralized leader. Even followers will be stressed because they will constantly receive updates
from the Zab leader.


\section{Results and Evaluation}
Figures 1 through 6 show the latencies of the two systems under each of the nine tasks and
experiments. Figure 7 shows the average, 99\%, and 99.9\% latencies for the mixed conflicting
read-write task of each experiment.

The results show that ZooKeeper has much lower latency for reads than writes. In addition, read
latency during conflicting writes is significantly higher than Safari. We believe that ZooKeeper's
read latency could be improved through the use of UDP instead of TCP, C++ instead of Java, and with
clients sending requests to multiple servers and awaiting the first response rather than always
using the same server.

We can see from Figure 7 that tail latency is fairly good in both systems. In both real world
settings, ZooKeeper's 99.9\% latency is no more than 2x its average latency. Safari has must lower
tail latency in the two datacenter deployment because read requests can complete successfully with
no round trips outside of the west coast datacenter. Even in the three datacenter settings, Safari
has low tail latency, only $\approx 5\%$ greater than the average latency. This is because reads
complete successfully as soon as the client receives any two responses, so slow responses from any
one data center do not matter.

Figure 7 also shows the surprising result that ZooKeeper had worse latency in the two datacenter
deployment than in the three datacenter deployment. We believe this is probably caused by the west
coast client accessing the east coast server for reads, although this claim should be investigated
further.

We claimed that Safari was designed to minimize latency. Indeed, the Figures clearly show that
Safari has more consistent and much lower tail latency than ZooKeeper. This is not surprising
considering the Safari's shortcomings. However, the local experiment in particular demonstrates that
ZooKeeper's latency could be drastically improved by changes to the implementation.

We also found that ZooKeeper performed inconsistently across runs of the same experiment. Typically
each run would take just a minute or two. Roughly one third of the time, ZooKeeper would crawl at a
pace such that it would have taken almost an hour to finish the experiment. This may have been
caused by clients selecting distant servers from which to read, and could probably have been
resolved through reconfiguration, tuning, or with different client software. However, we were
surprised that performance was so inconsistent in a seemingly typical deployment.

Overall we found ZooKeeper deployment to be fairly simple. All of the code necessary to download and
run ZooKeeper on EC2 can be found in roughly 5 lines of shell script and 15 lines of ZooKeeper
configuration. Although we had to manually tell each ZooKeeper server the IP addresses of all other
servers, this could be made less painful by spending more and purchasing long-lived IPs instead of
the transient ones we used, or using managed DNS. Safari was of course also easy to configure.
Athough the current implementation accepts a list of peer servers, servers never send messages to
one another. The only necessary configuration is to choose a UDP port on which the servers listen
for messages.

\begin{figure}[h]
  \caption{Zookeeper Latency on a Single Machine}
  \centering
  \includegraphics[width=0.5\textwidth]{../results/local/zookeeper.pdf}
\end{figure}

\begin{figure}[h]
  \caption{Safari Latency on a Single Machine}
  \centering
  \includegraphics[width=0.5\textwidth]{../results/local/safari.pdf}
\end{figure}


\begin{figure}[h]
  \caption{Zookeeper Latency Distributed Across Two Datacenters}
  \centering
  \includegraphics[width=0.5\textwidth]{../results/exp1/zookeeper.pdf}
\end{figure}

\begin{figure}[h]
  \caption{Safari Latency Distributed Across Two Datacenters}
  \centering
  \includegraphics[width=0.5\textwidth]{../results/exp1/safari.pdf}
\end{figure}

\begin{figure}[h]
  \caption{Zookeeper Latency Distributed Across Three Datacenters}
  \centering
  \includegraphics[width=0.5\textwidth]{../results/exp2/zookeeper.pdf}
\end{figure}

\begin{figure}[h]
  \caption{Safari Latency Distributed Across Three Datacenters}
  \centering
  \includegraphics[width=0.5\textwidth]{../results/exp2/safari.pdf}
\end{figure}


\begin{figure}[h]
  \caption{Mixed Read-Write Latency Statistics (ms)}
  \begin{center}
    \begin{tabular}{| l | l | l | l |}
    \hline
    & Average & 99\% & 99.9\% \\ \hline
    Local ZooKeeper & 13.414 & 26.090 & 33.073 \\ \hline
    Local Safari & 0.307 & 0.441 & 1.341 \\ \hline\hline
    2 Datacenter ZooKeeper & 148.230 & 272.393 & 279.877 \\ \hline
    2 Datacenter Safari & 0.728 & 1.052 & 1.439 \\ \hline\hline
    3 Datacenters ZooKeeper & 76.873 & 93.481 & 94.716 \\ \hline
    3 Datacenters Safari & 45.981 & 46.391 & 47.748 \\
    \hline
    \end{tabular}
  \end{center}
\end{figure}


%\section{Conclusion}
%\input{conclusion.tex}

% Put references on a page by themselves when using endfloat and the
% captionsoff option.
\newpage


% trigger a \newpage just before the given reference
% number - used to balance the columns on the last page
% adjust value as needed - may need to be readjusted if
% the document is modified later
%\IEEEtriggeratref{8}
% The "triggered" command can be changed if desired:
%\IEEEtriggercmd{\enlargethispage{-5in}}
\nocite{*}
\setlength\parindent{24pt}
\def\bibindent{1em}
\begin{thebibliography}{99\kern\bibindent}
  \bibitem{1}
  \texttt{Hunt et. al. "ZooKeeper: Wait-free coordination for Internet-scale systems".
  dl.acm.org/citation.cfm?id=1855851}

  \bibitem{2}
  \texttt{DeCandia et. al. "Dynamo: Amazon’s Highly Available Key-value Store".
  dl.acm.org/citation.cfm?id=1294281}

  \bibitem{3}
  \texttt{Dean, Barroso. "The Tail at Scale". dl.acm.org/ft\_gateway.cfm?id=2408794}

  \bibitem{4}
  \texttt{Dragojević et. al. "No compromises: distributed transactions with consistency, availability, and performance"
  dl.acm.org/citation.cfm?id=2815425}

  \bibitem{5}
  \texttt{Apache Kafka. https://kafka.apache.org/}

  \bibitem{6}
  \texttt{Patrick Hunt. "ZooKeeper service latencies under various loads \& configurations"
  https://wiki.apache.org/hadoop/ZooKeeper/ServiceLatencyOverview}

  \bibitem{7}
  \texttt{http://grokbase.com/t/kafka/users/1523ht96m5/kafka-long-tail-latency-issue}

  \bibitem{8}
  \texttt{https://drive.google.com/open?id=1PEm2sHj1Vokx812VfvHQP3s9Vy1TacWb}

  \bibitem{9}
  \texttt{https://github.com/mgraczyk/cs244b-project}

\end{thebibliography}

% references section
%\bibliographystyle{IEEEtran}
%\bibliography{bibliography}

% You can push biographies down or up by placing
% a \vfill before or after them. The appropriate
% use of \vfill depends on what kind of text is
% on the last page and whether or not the columns
% are being equalized.

%\vfill

% Can be used to pull up biographies so that the bottom of the last one
% is flush with the other column.
%\enlargethispage{-5in}


\end{document}
